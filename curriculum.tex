\documentclass[11pt,a4paper,sans,unicode]{moderncv}
\moderncvstyle{casual}                  
\moderncvcolor{blue}                             
\usepackage[utf8]{inputenc}                   
\usepackage[unicode]{hyperref}
\hypersetup{
    colorlinks=true,
    linkcolor=blue,
    filecolor=magenta,      
    urlcolor=cyan,
}
 
\urlstyle{same}
\usepackage[scale=0.75]{geometry}
\usepackage{import}
\name{Erick}{Flejan}
\title{Curriculum Vitae}                  
\address{Parroquia Santa Rosalía}{Caracas, Venezuela.}{}
\phone[mobile]{+58 426-316-5212}                   
\email{erickflejan67@gmail.com}                       
\social[github]{esft24}
\photo[52pt][0.2pt]{picturillo}                 

\makeatletter
\renewcommand*{\bibliographyitemlabel}{\@biblabel{\arabic{enumiv}}}
\makeatother

%----------------------------------------------------------------------------------
%            content
%----------------------------------------------------------------------------------
\begin{document}
\makecvtitle

\small{Estudiante de 5to año de Ingeniería de la Computación, actualmente cursando el treceavo trimestre en la Universidad Simón Bolívar. Apasionado con los desafíos que involucran algoritmos y codificación. Con experiencia realizando proyectos de desarrollo de software y sistemas utilizando metodologías ágiles y documentadas de desarrollo.}

\section{Educación}
\begin{itemize}
\vspace{2pt}
\item \textbf{Universidad Simón Bolívar:} \emph{2012 - Actualidad} \\ Actualmente cursando la carrera de Ingeniería en Computación. \\ \emph{Promedio: 3.959/5} 
\vspace{2pt}
\end{itemize}

\section{Habilidades Técnicas}

\vspace{2pt}

\begin{itemize}

\item \textbf{Lenguajes de Programación:} Competente en: Haskell, Scala, Ruby, C, Python, Matlab, \LaTeX, JavaScript. \\ También habilidades básicas con: Assembler/MIPS, Prolog, R, Bash.

\vspace{2pt}

\item \textbf{Frameworks y Desarrollo Web:} Competente en: Framework Django, Framework Web2py, conocimientos en HTML5 para desarrollo de front-end junto con CSS, Bootstrap4 y JavaScript como complemento.  

\vspace{2pt}

\item \textbf{Manejo de Bases de Datos y Otras Herramientas:} Competente en: PostgreSQL, SQLite. \\ También habilidades básicas con: Bases de Datos No-SQL como Jena. Entorno de ejecución Nodejs.

\end{itemize}

\section{Actividades Extra-Curriculares y Certificaciones}

\begin{itemize}
\item{Certificación de conocimiento en principios básicos de programación funcional y diseño funcional-reactivo. \href{https://www.coursera.org/account/accomplishments/certificate/2MHZ3UG58EHF}{Certificación puede ser verificada aquí.}. \emph{2018}
}
\item{Certificación de finalización del Programa de Técnicas de Lectura Integral y Selectiva constado por Técnicas Americanas de Estudio. \emph{2007}}
\end{itemize}

\section{Proyectos}

\begin{itemize}

\item{\textbf{Avanzómetro}. Aplicación web con fines de visualización estadística de las notas de los estudiantes de la Universidad Simón Bolívar. Realizada en conjunto en el equipo de desarrollo \emph{Athenas Development} utilizando Django como framework y Scrum como metodología de desarrollo. Realizado para el curso de \emph{Ingeniería de Software I}.}

\item{\textbf{Módulo del Sistema de Gestión Integral de la Unidad de Laboratorios (SIGULAB)}. Realizado en conjunto en el equipo de desarrollo \emph{Octopus Technologies} utilizando Web2py como framework. Realizado para el curso de \emph{Sistemas de Información I}.}

\item{\textbf{CodeyBoy (Lenguaje de Programación)}. Lenguaje de programación fuertemente tipado de uso general. Realizado para el curso de \emph{Lenguajes de Programación 2 y 3}}

\item{\textbf{Sistema MECE} \emph{Mecanismo para el Empoderamiento de Competencias Educativas}. Sistema de admisión al estudio en la Universidad Simón Bolívar,
diseñado para que el participante se haga consciente de su propia formación
con el fin de que pueda fortalecer sus competencias educativas.}

\end{itemize}
\nocite{*}
\bibliographystyle{plain}
\bibliography{publications}            
\end{document}
